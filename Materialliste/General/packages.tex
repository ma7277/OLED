%%%%%%%%%%%%%%%%%%%%%%%%%%%%%%
%
% $Autor: Wings $
% $Datum: 2020-01-30 07:43:10Z $
% $Pfad: komponenten/Bilderkennung/Produktspezifikation/JetsonNano/Allgemein/packages.tex $
% $Version: 1786 $
%
%
%%%%%%%%%%%%%%%%%%%%%%%%%%%%%%




\usepackage[left=4cm,right=4cm,top=2cm,bottom=2cm,includeheadfoot]{geometry}
\setlength{\headheight}{1pt}

% Standard Deutsch
\usepackage{lmodern}
\usepackage[utf8]{inputenc}
\ifdefined\isGerman
  \usepackage[ngerman]{babel}
\else
  \ifdefined\isEnglish
    \usepackage[english]{babel}
  \else
    \usepackage[ngerman]{babel}
  \fi
\fi

\usepackage[T1]{fontenc}

% Farben
\usepackage{xcolor}
\usepackage{alltt}
% Notwendig für Syntax-Highlighting mit besonderen Farben
%\usepackage[table, x11names]{xcolor}
\usepackage{color}
\usepackage{soul}

% Pakete für Mathe
\usepackage{amsmath}
\usepackage{amsfonts}
\usepackage{amssymb}
\usepackage{mathtools}


% Paket Internetrefferenzen
\usepackage{hyperref}

%Wird zusätzlich das Paket showidx eingebunden, werden die gesetzten Einträge am Seitenrand dargestellt. 
\usepackage{imakeidx}
%\usepackage{showidx}
%\usepackage[override=false]{seealso}
\makeindex[intoc,title=Index]
\ifdefined\isGerman
  \renewcommand{\indexname}{Stichwortverzeichnis}
\else
\ifdefined\isEnglish
\else
  \renewcommand{\indexname}{Stichwortverzeichnis}
\fi
\fi




% Paktete für Bilder
\usepackage{graphicx}

\usepackage{listings}
% Automatische erstellung von Referenzen
\usepackage[ngerman]{cleveref}
% Zeilenumbrüche in \texttt
\usepackage[htt]{hyphenat}

% Automatisches erstellen und referenzieren eines Abkürzungsverzeichnisses
\usepackage[tooltip]{acro}
% Ermöglicht Verwendung von 'H' Float modifier
\usepackage{float}

% Ermöglicht Subfigures und Subcaptions
\usepackage{subcaption}


\usepackage{float}
\usepackage{array}
\usepackage{amsopn}
\usepackage[percent]{overpic}
\usepackage{colortbl}
\usepackage{marginnote}

\usepackage{textcomp}

\usepackage[
backend=bibtex8,%  defernumbers=true,
autocite=inline,
labelalpha=true,%  sorting=none,
firstinits=true,
uniquename=init,
uniquelist=false,
refsegment=section,
style=alphabetic,
safeinputenc
]{biblatex}

%\usepackage[german]{babelbib}
%\usepackage[square,sort,comma,numbers]{natbib}
%\DeclareNameAlias{sortname}{last-first}
%\bibliographystyle{geralpha}
%\setcitestyle{square,aysep={},yysep={;}}
%\newcommand*{\multicitedelim}{\addsemicolon\space}


% Verwendung:
%\Ausblenden{.....
%} %todo Ausblenden



% Kapitel entfernen
%\usepackage{titlesec}

\usepackage{tabularx} 
\usepackage{longtable}
\usepackage{framed}
\usepackage{tocbibind}

\usepackage[official]{eurosym}

\definecolor{mygreen}{rgb}{0,0.6,0}
\definecolor{mygray}{rgb}{0.5,0.5,0.5}
\definecolor{mymauve}{rgb}{0.58,0,0.82}


%\usepackage{pythonhighlight}


\pagestyle{empty}




% Kopfzeile
\usepackage{fancyhdr}
\pagestyle{fancy}
\fancyhf{}
\fancyhead[OL]{\nouppercase{\leftmark}}
\fancyhead[OR]{\thepage}
\fancyhead[ER]{\nouppercase{\leftmark}}
\fancyhead[ER]{\nouppercase{\chaptermark{}}}
\fancyhead[ER]{\nouppercase{\chaptermark{}}}
\fancyhead[EL]{\thepage}
\renewcommand{\headrulewidth}{0.1pt} 



%\AtBeginDocument{\renewcommand{\chaptername}{}} % muss nach babel stehen! 



\usepackage{ifthen}

\usepackage{media9}
\usepackage{animate}
\newcounter{AnimateAngle}
\setcounter{AnimateAngle}{0}
\newcounter{AnimateStepFactor}
\setcounter{AnimateStepFactor}{1}
\newcounter{AnimateSteps}
\setcounter{AnimateSteps}{0}


\usepackage{multirow}                  % Tabelle vertikale Zellen verbinden
\usepackage{rotating}                  % text rotieren \rotatebox{90}{text}
\usepackage{colortbl}                  % Farbe der Zellen einer Tabelle, z. B. \begin{tabular}{c>{\columncolor{green}}cc}

\usepackage{enumitem}
\usepackage{spverbatim}

% Korrekte darstellung von SI-Einheiten
\usepackage[binary-units=true, per-mode=symbol]{siunitx}
% Erstellen einer ToDo-Liste
\usepackage[ngerman, colorinlistoftodos]{todonotes}
% Kommando zum markieren fehlender praktischer Umsetzungen
\newcommand{\todop}[2][]{\todo[color=red, inline, #1]{#2}}
\usepackage{makecell}
% Bera Mono als \ttfamily font
\usepackage[scaled]{beramono}

\usepackage{menukeys}
\usepackage{textcomp}


% Add section names to todonotes list
\makeatletter
\let\ori@chapter\@chapter
\def\@chapter[#1]#2{\ori@chapter[#1]{#2}%
    \if@mainmatter\addcontentsline{tdo}{chapter}{\protect\numberline{\thechapter}{#1}}%
    \else\addcontentsline{tdo}{chapter}{#1}%
    \fi}
\makeatother

% Farben für Syntax-Highlighting
\definecolor{dkgreen}{rgb}{0,.6,0}
\definecolor{dkblue}{rgb}{0.655,0.113,.364}
\definecolor{dkyellow}{cmyk}{0,0,.8,.3}

\definecolor{parameterc}{rgb}{.4,0,.6}
\definecolor{typec}{rgb}{0,0.525,.702}
\definecolor{stringc}{rgb}{0,.5019,.5019}
\definecolor{keywordc}{rgb}{.6549, .1137, .3647}
\definecolor{commentc}{rgb}{.5882, .5960, .5882}
\definecolor{textc}{rgb}{.2,.2,.2}

\lstdefinestyle{all}{
    alsoletter={-},
    frame=single, 	% top,frame=bottom,
    numbers=none,
    numberstyle=\tiny\color{textc},
    basicstyle=\linespread{0.9}\ttfamily\footnotesize\color{textc},
    tabsize=4,
    showstringspaces=false,
    captionpos=t,
    rulecolor=\color{lightgray!40},
    keywordstyle=\color{keywordc},
    stringstyle=\color{stringc},
    commentstyle=\color{commentc},
    breaklines=true,
    escapechar="!",
    postbreak=\mbox{\textcolor{green}{$\hookrightarrow$}\space},
}

\lstdefinestyle{bashstyle}{
    style=all,
    keywords=[2]{-y, --no-install-recommends, --allow-change-held-packages, --allow-downgrades, --fetch-keys, -n, --version, --params, -c, -i, -O, --upgrade, --no-cache-dir, --extra-index-url, --show, -s, -m},
    keywordstyle=[2]\color{parameterc},
    morekeywords = {ln,choco,pip,pip3,apt,apt-key,apt-get,apt-mark,add-apt-repository,wget,mktemp,dpkg,dpkg-query,echo,>>,rm,tegrastats, systemctl},
    deletekeywords={local,LOCAL},
}

\lstdefinestyle{pythonstyle}{
    style=all,
    morekeywords={as},
    keywords=[2]{True, False, None},
    keywordstyle=[2]\color{typec},
    alsoletter={_},
    keywords=[3]{max_workspace_size_bytes, precision_mode, maximum_cached_engines, use_calibration, optimizer, loss, input_shape, from_logits, metrics, batch_size, epochs, validation_data, activation, use_calibration, filters, kernel_size, pool_size, units},
    keywordstyle=[3]\color{parameterc},
    deletekeywords={compile,COMPILE},
}

\lstdefinestyle{inlinestyle}{
    style=all,
    breaklines        = true,
    breakatwhitespace = true,
    breakindent       = 2ex,
    escapechar        = *,
    numbers           = left,
    postbreak=,
}
\lstdefinelanguage{MyBash} {
    language = Bash,
    style=bashstyle,
}

\lstdefinelanguage{MyPython} {
    language = Python,
    style=pythonstyle,
}

\usepackage[intoc]{nomencl}
\renewcommand{\nomname}{Symbolverzeichnis}
\renewcommand{\nomlabel}[1]{#1 \dotfill}
\makenomenclature

\usepackage{enumitem,amssymb}
\newlist{todolist}{itemize}{2}
\setlist[todolist]{label=$\square$}

\usepackage{blindtext}